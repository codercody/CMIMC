\documentclass[10pt]{article}
\usepackage{amsmath, amssymb, amsthm, mathtools, enumerate}
\usepackage[top=2cm, left = 2cm, right = 2cm, bottom = 3cm]{geometry}
\usepackage[pdftex]{graphicx}
\usepackage{asymptote}
\usepackage{fancyhdr}
\newcommand{\N}{\mathbb{N}}
\newcounter{enum}
\setcounter{enum}{0}
\pagestyle{fancy}
\rhead{}
\chead{\includegraphics[scale=0.17]{CMIMC-header-2017.png}}
\lhead{}
\setlength{\headheight}{43pt}
\rfoot{}
\cfoot{}
\lfoot{}
\newcommand{\proposed}[1]
{
\vspace{5pt}
\noindent\textit{Proposed by #1}
}
\newcommand{\solution}
{
\vspace{5pt}
\noindent\textit{Solution.}\qquad
}
\DeclarePairedDelimiter\abs{\lvert}{\rvert}
\begin{document}

\begin{center}
\huge\textbf{Algebra Tiebreaker}
\end{center}

\begin{enumerate}
	\item Find all real numbers $x$ such that the expression
\[\log_2 \abs{1 + \log_2 \abs{2 + \log_2 \abs{x}}}\]
does not have a defined value.
	
	\item Let $x$ be a real number between $0$ and $\tfrac{\pi}2$ such that \[\dfrac{\sin^4(x)}{42}+\dfrac{\cos^4(x)}{75} = \dfrac{1}{117}.\] Find $\tan(x)$.
	
	\item The parabola $\mathcal P$ given by equation $y=x^2$ is rotated some acute angle $\theta$ clockwise about the origin such that it hits both the $x$ and $y$ axes at two distinct points.  Suppose the length of the segment $\mathcal P$ cuts the $x$-axis is $1$.  What is the length of the segment $\mathcal P$ cuts the $y$-axis?
	
\end{enumerate}

\newpage

\begin{center}
\huge\textbf{Combinatorics Tiebreaker}
\end{center}

\begin{enumerate}

\item Jesse has ten squares, which are labeled $1, 2, \dots, 10$. In how many ways can he color each square either red, green, yellow, or blue such that for all $1 \le i < j \le 10$, if $i$ divides $j$, then the $i$-th and $j$-th squares have different colors?

\item Kevin likes drawing. He takes a large piece of paper and draws on it every rectangle with positive integer side lengths and perimeter at most 2017, with no two rectangles overlapping. Compute the total area of the paper that is covered by a rectangle.

\item In a certain game, the set $\{1, 2, \dots, 10\}$ is partitioned into equally-sized sets $A$ and $B$. In each of five consecutive rounds, Alice and Bob simultaneously choose an element from $A$ or $B$, respectively, that they have not yet chosen; whoever chooses the larger number receives a point, and whoever obtains three points wins the game. Determine the probability that Alice is guaranteed to win immediately after the set is initially partitioned.

\end{enumerate}

\newpage

\begin{center}
\huge\textbf{Computer Science Tiebreaker}
\end{center}

\begin{enumerate}

\item Cody has an unfair coin that flips heads with probability either $\tfrac13$ or $\tfrac23$, but he doesn't know which one it is. Using this coin, what is the fewest number of independent flips needed to simulate a coin that he knows will flip heads with probability $\tfrac13$?


\item Define
\[f(h,t) =
\begin{cases}
8h & h = t \\
(h-t)^2 & h \neq t.
\end{cases}\]
Cody plays a game with a fair coin, where he begins by flipping it once. At each turn in the game, if he has flipped $h$ heads and $t$ tails and $h + t < 6$, he can choose either to stop and receive $f(h,t)$ dollars or he can flip the coin again; if $h + t = 6$ then the game ends and he receives $f(h,t)$ dollars. If Cody plays to maximize expectancy, how much money, in dollars, can he expect to win from this game? 


\item Let $n=2017$ and $x_1,\dots,x_n$ be boolean variables. An \emph{$7$-CNF clause} is an expression of the form $\phi_1(x_{i_1})+\dots+\phi_7(x_{i_7})$, where $\phi_1,\dots,\phi_7$ are each either the function $f(x)=x$ or $f(x)=1-x$, and $i_1,i_2,\dots,i_7\in\{1,2,\dots,n\}$. For example, $x_1+(1-x_1)+(1-x_3)+x_2+x_4+(1-x_3)+x_{12}$ is a $7$-CNF clause. What's the smallest number $k$ for which there exist $7$-CNF clauses $f_1,\dots,f_k$ such that \[f(x_1,\dots,x_n):=f_1(x_1,\dots,x_n)\cdots f_k(x_1,\dots,x_n)\] is $0$ for all values of $(x_1,\dots,x_n)\in\{0,1\}^n$?

\end{enumerate}

\newpage

\begin{center}
\huge\textbf{Geometry Tiebreaker}
\end{center}

\begin{enumerate}

\item Let $ABCD$ be an isosceles trapezoid with $AD\parallel BC$.  Points $P$ and $Q$ are placed on segments $\overline{CD}$ and $\overline{DA}$ respectively such that $AP\perp CD$ and $BQ\perp DA$, and point $X$ is the intersection of these two altitudes.  Suppose that $BX=3$ and $XQ=1$.  Compute the largest possible area of $ABCD$.

\item Points $A$, $B$, and $C$ lie on a circle $\Omega$ such that $A$ and $C$ are diametrically opposite each other.  A line $\ell$ tangent to the incircle of $\triangle ABC$ at $T$ intersects $\Omega$ at points $X$ and $Y$.  Suppose that $AB=30$, $BC=40$, and $XY=48$.  Compute $TX\cdot TY$.

\item Triangle $ABC$ satisfies $AB=104$, $BC=112$, and $CA=120$.  Let $\omega$ and $\omega_A$ denote the incircle and $A$-excircle of $\triangle ABC$, respectively.  There exists a unique circle $\Omega$ passing through $A$ which is internally tangent to $\omega$ and externally tangent to $\omega_A$.  Compute the radius of $\Omega$.

\end{enumerate}

\newpage

\begin{center}
\huge\textbf{Number Theory Tiebreaker}
\end{center}

\begin{enumerate}
	\item Let $\tau(n)$ denote the number of positive integer divisors of $n$. For example, $\tau(4) = 3$. Find the sum of all positive integers $n$ such that $2 \tau(n) = n$.
	
	\item Find the smallest three-digit divisor of the number \[1\underbrace{00\ldots 0}_{100\text{ zeros}}1\underbrace{00\ldots 0}_{100\text{ zeros}}1.\]
	
	\item Say an integer polynomial is \textit{primitive} if the greatest common divisor of its coefficients is $1$.  For example, $2x^2+3x+6$ is primitive because $\gcd(2,3,6)=1$.  Let $f(x)=a_2x^2+a_1x+a_0$ and $g(x) = b_2x^2+b_1x+b_0$, with $a_i,b_i\in\{1,2,3,4,5\}$ for $i=0,1,2$.  If $N$ is the number of pairs of polynomials $(f(x),g(x))$ such that $h(x) = f(x)g(x)$ is primitive, find the last three digits of $N$.
	
\end{enumerate}

\end{document}