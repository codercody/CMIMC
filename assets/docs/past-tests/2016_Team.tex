\documentclass[10pt]{article}
\usepackage{amsmath, amssymb, amsthm}
\usepackage[top=2cm, left = 2cm, right = 2cm, bottom = 3cm]{geometry}
\usepackage[pdftex]{graphicx}
\usepackage{asymptote}
\usepackage{fancyhdr}
\pagestyle{fancy}
\rhead{}
\chead{\includegraphics[scale=0.12]{CMIMC-header.png}}
\lhead{}
\setlength{\headheight}{43pt}
\rfoot{}
\cfoot{}
\lfoot{}
\addtolength\footskip{-1cm}
\begin{document}\thispagestyle{empty}
\begin{center}

\vspace*{90pt}

\includegraphics[scale=0.2]{CMIMC-header.png}

\includegraphics[scale=0.35]{team-header.png}

\vspace{1.6in}

\includegraphics[scale=0.20]{instruction-header.png}
\noindent\rule{17.7cm}{2pt}
\end{center}

\vspace{10pt}

\begin{enumerate}
	\large
	\item Do not look at the test before the proctor starts the round.
	\item This test consists of 10 short-answer problems to be solved in 30
		minutes.
	\item Write your team name and team ID on your answer sheet.
	\item Write your answers in the corresponding boxes on the answer
		sheets.
	\item No computational aids other than pencil/pen are permitted.
	\item All answers are integers.
	\item If you believe that the test contains an error, submit your
		protest in writing to Porter 100.
\end{enumerate}

\newpage

\begin{center}
\huge\textbf{Team}\normalsize

\vspace{3pt}
\end{center}

\begin{enumerate}
\setlength{\itemsep}{3pt}

%\item Let $\tau(n)$ denote the number of positive divisors of $n$.  For example, $\tau(6)=4$ since the divisors of $6$ are $1$, $2$, $3$, and $6$.  What is the value of \[1\cdot (-1)^{\tau(1)}+2\cdot (-1)^{\tau(2)}+3\cdot (-1)^{\tau(3)}+\cdots+100\cdot (-1)^{\tau(100)}?\]

\item Construction Mayhem University has been on a mission to expand and improve
	its campus!  The university has recently adopted a new construction
	schedule where a new project begins every two days. Each project will take exactly one more day than the previous one to complete (so the first project takes 3, the second takes 4, and so on.)

\par Suppose the new schedule starts on Day 1.  On which day will there first be at least $10$ projects in place at the same time?

\item Right isosceles triangle $T$ is placed in the first quadrant of the coordinate plane.  Suppose that the projection of $T$ onto the $x$-axis has length $6$, while the projection of $T$ onto the $y$-axis has length $8$.  What is the sum of all possible areas of the triangle $T$?

\begin{figure}[ht]
	\centering
	\begin{asy}
	import olympiad;
	size(120);
	defaultpen(linewidth(0.8));
	pair A = (0.9,0.6), B = (1.7, 0.8), C = rotate(270, B)*A;
	pair PAx = (A.x,0), PBx = (B.x,0), PAy = (0, A.y), PCy = (0, C.y);
	draw(PAx--A--PAy^^PCy--C^^PBx--B, linetype("4 4"));
	draw(rightanglemark(A,B,C,3));
	draw(A--B--C--cycle);
	draw((0,2)--(0,0)--(2,0),Arrows(size=8));
	label("$6$",(PAx+PBx)/2,S);
	label("$8$",(PAy+PCy)/2,W);
\end{asy}
\end{figure}

\item We have 7 buckets labelled 0-6. Initially bucket 0 is empty, while
	bucket $n$ (for each $1 \leq n \leq 6$) contains the list $[1,2,
	\ldots, n]$. Consider the following program: choose a subset $S$ of
	$[1,2,\ldots,6]$ uniformly at random, and replace the contents of
	bucket $|S|$ with $S$. Let $\frac{p}{q}$ be the probability that bucket
	5 still contains $[1,2, \ldots, 5]$ after two executions of this
	program, where $p,q$ are positive coprime integers. Find $p$. %ans: 305

\item For some integer $n > 0$, a square paper of side length $2^{n}$ is
	repeatedly folded in half, right-to-left then bottom-to-top, until a
square of side length 1 is formed. A hole is then drilled into the square at a
point $\tfrac{3}{16}$ from the top and left edges, and then the paper is
completely unfolded. The holes in the unfolded paper form a rectangular array of
unevenly spaced points; when connected with horizontal and vertical line
segments, these points form a grid of squares and rectangles. Let $P$ be a point
chosen randomly from \textit{inside} this grid. Find the largest $L$ such that,
for all $n$, the probability that the four segments $P$ is bounded by form a
square is at least $L$. %patrick

\item Recall that in any row of Pascal's Triangle, the first and last elements of the row are $1$ and each other element in the row is the sum of the two elements above it from the previous row.  With this in mind, define the \textit{Pascal Squared Triangle} as follows:

\begin{itemize}

\item In the $n^{\text{th}}$ row, where $n\geq 1$, the first and last elements of the row equal $n^2$;

\item Each other element is the sum of the two elements directly above it.

\end{itemize}

The first few rows of the Pascal Squared Triangle are shown below.

\[\begin{array}{c@{\hspace{7em}}
c@{\hspace{2pt}}c@{\hspace{2pt}}c@{\hspace{4pt}}c@{\hspace{2pt}}
c@{\hspace{2pt}}c@{\hspace{2pt}}c@{\hspace{2pt}}c@{\hspace{3pt}}c@{\hspace{2pt}}
c@{\hspace{2pt}}c} \vspace{4pt}
\text{Row 1: } & & & & & & 1 & & & & & \\\vspace{4pt}
\text{Row 2: } & & & & & 4 & & 4 & & & & \\\vspace{4pt}
\text{Row 3: } & & & & 9 & & 8 & & 9 & & & \\\vspace{4pt}
\text{Row 4: } & & &16& &17& &17& & 16& & \\\vspace{4pt}
\text{Row 5: } & &25 & &33& &34 & &33 & &25 &
\end{array}\]
Let $S_n$ denote the sum of the entries in the $n^{\text{th}}$ row.  For how many integers $1\leq n\leq 10^6$ is $S_n$ divisible by $13$? %David Altizio

\item Suppose integers $a < b < c$ satisfy \[ a + b + c = 95\qquad\text{and}\qquad a^2 + b^2 + c^2 = 3083.\] Find $c$.

\item In $\triangle ABC$, $AB=17$, $AC=25$, and $BC=28$.  Points $M$ and $N$ are the midpoints of $\overline{AB}$ and $\overline{AC}$ respectively, and $P$ is a point on $\overline{BC}$.  Let $Q$ be the second intersection point of the circumcircles of $\triangle BMP$ and $\triangle CNP$.  It is known that as $P$ moves along $\overline{BC}$, line $PQ$ passes through some fixed point $X$.  Compute the sum of the squares of the distances from $X$ to each of $A$, $B$, and $C$. %David Altizio

\item Let $N$ be the number of triples of positive integers $(a,b,c)$ with $a\leq b\leq c\leq 100$ such that the polynomial \[P(x)=x^2+(a^2+4b^2+c^2+1)x+(4ab+4bc-2ca)\] has integer roots in $x$.  Find the last three digits of $N$.

\item For how many permutations $\pi$ of $\{1,2,\ldots,9\}$ does there exist an integer $N$ such that \[N\equiv \pi(i)\pmod{i}\text{ for all integers }1\leq i\leq 9?\] %David Altizio

\item Let $\mathcal{P}$ be the unique parabola in the $xy$-plane which is tangent to the $x$-axis at $(5,0)$ and to the $y$-axis at $(0,12)$.  We say a line $\ell$ is $\mathcal{P}$-friendly if the $x$-axis, $y$-axis, and $\mathcal{P}$ divide $\ell$ into three segments, each of which has equal length.  If the sum of the slopes of all $\mathcal{P}$-friendly lines can be written in the form $-\tfrac mn$ for $m$ and $n$ positive relatively prime integers, find $m+n$. %David Altizio

\end{enumerate}

\end{document}

\newpage

We first make use of a lemma.

\par\textbf{LEMMA: }Let $A$ and $B$ be the points of tangency of the parabola with the $x$ and $y$ axes and $O$ the origin.  Then the focus of $\mathcal{P}$ is the projection of $O$ onto $AB$.

\par\textit{Proof. } Recall the reflection property of parabolas, which states that if $F$ is the focus of a parabola, then the reflection of $F$ about any tangent line lies on the directrix of the parabola.  With this in mind, let the reflections of $F$ about the $x$ and $y$ axes be $F_x$ and $F_y$ respectively.  Remark that by similar triangles $F_xF_y$ passes through $O$.  Furthermore, note that by the reflection property $FA=AF_x$.  But by definition a parabola is the set of points which are equidistant from the focus to the directrix, so $AF_x$ is also the distance from $A$ to $F_xF_y$, i.e. $AF_x\perp F_xF_y$.  By symmetry, therefore, $OF\perp FA$.  Similarly, $OF\perp FB$ and the result follows.

\par Unfortunately, the CMIMC staff does not have a more elegant solution past this point - we apologize for the inconvenience.  

\par Generalizing the problem, let $(a,b)$ be the focus of the parabola.  Then as a corollary of the above lemma, the directrix of the parabola is $y=-\tfrac bax$.  Now according to the Distance from Point to Line formula, we get that a point $(x,y)$ lies on $\mathcal{P}$ iff \[\sqrt{(x-a)^2+(y-b)^2}=\dfrac{|bx+ay|}{\sqrt{a^2+b^2}}\implies (x-a)^2+(y-b)^2=\dfrac{(bx+ay)^2}{a^2+b^2}.\] Now note that the condition of $\mathcal{P}$-friendliness equates to the existence of positive real numbers $x_0$ and $y_0$ such that $(x_0,2y_0)\in\mathcal{P}$ and $(2x_0,y_0)\in\mathcal{P}$.  As a result, it suffices to determine the sum of all possible values of $-\tfrac{y_0}{x_0}$ given that $x_0$ and $y_0$ both satisfy \[(x_0-a)^2+(2y_0-b)^2=\dfrac{(bx_0+2ay_0)^2}{a^2+b^2}\qquad\text{and}\qquad (2x_0-a)^2+(y_0-b)^2=\dfrac{(2bx_0+ay_0)^2}{a^2+b^2}.\] Note that subtracting the second equation from the first gives \[3(y_0^2-x_0^2)+2(ax_0-by_0)=\dfrac{3(a^2y_0^2-by_0^2)}{a^2+b^2}.\] Now since $(y_0^2-x_0^2)(a^2+b^2)=a^2y_0^2+b^2y_0^2-a^2x_0^2-b^2x_0^2$, we see that clearing denominators cancels several terms and leaves \[3(a^2x_0^2-b^2y_0^2)=2(ax_0-by_0)(a^2+b^2).\] Now if $ax_0=by_0$, then it's not hard to show that the original two quadratics are equivalent and thus there are two solutions to the original system - we leave this as an exercise to the reader.  Otherwise, we have $ax_0+by_0=\tfrac23(a^2+b^2)$.

\par Now adding the original two equations together gives \[5x_0^2-6x_0a+2a^2+5y_0^2-6y_0b+2b^2=\dfrac{5b^2x_0^2+8abx_0y_0+5a^2y_0^2}{a^2+b^2}.\] Note that the $ax_0+by_0$ expression appears here, and so we may substitute our known value for it to simplify the equation to \[5x_0^2+5y_0^2-2(a^2+b^2)=\dfrac{5b^2x_0^2+8abx_0y_0+5a^2y_0^2}{a^2+b^2}.\] Once again, multiply through by $a^2+b^2$ to clear denominators and simplify terms; this eventually gives \[5a^2x_0^2-8abx_0y_0+6y_0^2b^2=2(a^2+b^2)^2.\] Now squaring the equality $ax_0+by_0=\tfrac23(a^2+b^2)$ and subtracting yields $18abx_0y_0=\tfrac29(a^2+b^2)^2$, so $abx_0y_0=\tfrac1{81}(a^2+b^2)^2$.

\par Consider the system of equations \[ax_0+by_0=\dfrac23(a^2+b^2)\qquad\text{and}\qquad (ax_0)(by_0)=\dfrac1{81}(a^2+b^2)^2.\] Letting $X=3x_0(a^2+b^2)$ and $Y=3y_0(a^2+b^2)$ does not change the ratio $\tfrac{Y}{X}$ and additionally simplifies this system to \[aX+bY=2\qquad\text{and}\qquad (aX)(bY)=\frac19.\] It is easy to solve for $X$ and $Y$ from here and compute the sum of the two possible values for $\tfrac{Y}{X}$, but we can instead be somewhat clever and note that by Vieta's if $r_1$ and $r_2$ are the roots of the quadratic $t^2-2t+\tfrac19=0$, then the sum of the two possible values of $\tfrac{bY}{aX}$ must be \[\dfrac{r_1}{r_2}+\dfrac{r_2}{r_1}=\dfrac{r_1^2+r_2^2}{r_1r_2}=\dfrac{(r_1+r_2)^2}{r_1r_2}-2=34,\] so the sum of the (negatives of the) slopes of the two lines in this case is $34\tfrac{a}{b}$.

\par Overall, the sum of the (negatives of the) slopes of the four lines which work is \[34\cdot\dfrac{a}{b}+\dfrac{a}{b}+\dfrac{a}{b}=36\cdot\dfrac{a}{b}.\] Going back to the original problem, we see that since the line $OF$ is perpendicular to $AB$ its slope is $\tfrac{5}{12}$, meaning that $F$ lies on the line $y=\tfrac{5}{12}x$.  Therefore the sum of the slopes is $-36\cdot\tfrac{12}5=\tfrac{432}5$ and the requested answer is $\boxed{437}$.
\end{document}
